% macros <<<
\def\para #1.{\vskip0pt plus6pt\noindent{\it#1.\enspace}}
\abovedisplayskip=6pt plus 3pt minus 3pt
\abovedisplayshortskip=0pt plus 3pt
\belowdisplayskip=6pt plus 3pt minus 3pt
\belowdisplayshortskip=6pt plus 3pt minus 4pt

% >>>
This file contains notes written before the implementation of various tricky parts.
Notes are in chronological order (oldest first) and there is no requirement for a note to relate to previous ones.
This is mostly a historical document, so there is no promise that notes are kept up to date when the implementation changes.

\beginsection Non-unit transitions (201106) % <<<

The automata observe two types of events in the program, method calls and method returns.
Sometimes, however, it is convenient for the user to think of a (call, return) pair as an atomic event.
For example, a transition labeled $I:=C.{\it iterator}()$ refers to both the call and the return of the method {\it iterator}.
The user can tag a transition with {\it call\/} or {\it return\/} to say which type of event may trigger it.
If no tag is present then the transition is said to be a call--return transition.
This note describes how to deal with such transitions that consume more than one event.

Each call or return transition has a guard and an action:
The guard examines the event and decides if the transition should be performed; the action writes to the automaton memory.
$$x\to y:\; (g, a)  \eqno(1)$$
Intuitively, a call--return transition has two guards and two actions.
$$x\to y:\; (g_0,a_0); (g_1,a_1)  \eqno(2)$$
In general, we could allow a list of (guard, action) pairs.
$$x\to y: (g_0, a_0);\ldots;(g_{n-1},a_{n-1})  \eqno(3)$$
{\it Unit transitions\/} are those with a list of length~$1$.
Suppose the automaton is in state~$(x,\sigma_0)$, for some store~$\sigma_0$, and the stream of events is $e_0,e_1,e_2,\ldots$
We say that an outgoing transition of~$x$ {\it matches\/} when $g_k\;e_k\;\sigma_k$ holds for all~$k\in[\,0.\,.\,n)$, where $\sigma_{k+1}=a_k\;e_k\;\sigma_k$.
If none of the outgoing transitions of~$x$ match, then the automaton remains in the same state, $e_0$~is discarded, and the automaton continues to process the events $e_1,e_2,\ldots$
Otherwise, one of the matching transitions is picked nondeterministically.
The automaton {\it matches\/} the events $e_0,\ldots,e_{n-1}$ by going to the state~$(y,\sigma_n)$ and then continues to process the events $e_n,e_{n+1},\ldots$

Consider an implementation that follows closely the previous description.
Also consider an automaton with a single edge like~(3).
Suppose the stream of events is very long and the transition always fails because of the last guard.
We make a few observations.
\smallskip
\item- Each event will be examined $n$~times, once by each guard.
\item- The last $n$~events coming from the stream must be kept in memory.
\item- The state of the automaton lags behind the stream of events.
\smallskip\noindent
These observations suggest that the implementation is much simpler if it only handles the case~$n=1$.

Is there a desugaring to equivalent unit automata?
To illustrate the issues, let us try to desugar one call--return transition.
$$x\to y:\; (g_0,a_0); (g_1,a_1)  \eqno(4)$$
The first idea is to use two edges.
$$\eqalign{
  &x\to z:\; (g_0,a_0) \cr
  &z\to y:\; (g_1,a_1)
}\eqno(5)$$
Suppose that initially the automata are in state~$(x,\sigma)$ and the stream of events $e_0,e_1,e_2$ satisfies the following.
$$\eqalign{
  &g_0\;e_0\;\sigma \cr
  &\lnot(g_1\;e_1\;(a_0\;e_0\;\sigma)) \cr
  &g_0\;e_1\;\sigma \cr
  &g_1\;e_2\;(a_0\;e_1\;\sigma) \cr
  &g_1\;e_2\;(a_0\;e_0\;\sigma)
}\eqno(6)$$
Automaton~$(4)$ discards $e_0$ and then matches $e_1,e_2$ by going to~$(y,a_1\;e_2\;(a_0\;e_1\;\sigma))$.
But $(5)$~matches~$e_0$, discards~$e_1$, matches~$e_2$, and ends up in~$(y,a_1\;e_2\;(a_0\;e_0\;\sigma))$.

We considered three ideas for desugaring:
\smallskip
\item- Do all the actions at the end and rewrite guards.
\item- Use nondeterminism to keep around both the modified and the original store.
\item- Enhance the semantics with {\it undo\/} actions and use them.
\smallskip\noindent Let us consider them in turn.

The basic skeleton for the desugaring based on guard rewriting is the following.
$$\eqalign{
  &x\to z:\; (g_0,0) \cr
  &z\to y:\; (g'_1,a_0a_1)
}\eqno(7)$$
Here $g'_1$ is a guard derived from the information available in the initial automaton ($g_0$, $g_1$, $a_0$, $a_1$) and $a_0a_1$~is some composition of the actions.
We want $g'_1\;e\;\sigma=g_1\;e\;(a_0\;e'\;\sigma)$ for all stores~$\sigma$ and all events $e$~and~$e'$.
There is no such function, because only the right hand side depends on~$e'$.

The basic skeleton for the desugaring based on exploiting nondeterminism is the following.
$$\eqalign{
  &x\to z:\; (g_0, a_0) \cr
  &z\to y:\; (g_1, a_1) \cr
  &x\to w:\; (g_0, 0) \cr
  &w\to x:\; (\lnot g_1, 0)
}\eqno(8)$$
Here $0$~is a no-op and $\lnot g_1$ is a guard defined by $(\lnot g_1)\;e\;\sigma=\lnot(g_1\;e\;\sigma)$.
The intuition is that vertex~$z$ represents partial success and vertex~$w$ represents partial failure.
The transition~$w\to x$ is wrong because the guard~$g_1$ is evaluated on the original store.
Suppose this issue could be fixed, although I don't know how.
There is an even bigger problem.
Consider the example~$(6)$.
After $e_0$, the automaton nondeterministically proceeds to $z$~or~$w$.
(We can imagine it is in both states.)
After $e_1$, the automaton should be in state $(z,a_0\;e_1\;\sigma)$ and should leave vertex~$w$.
This is because $g_1\;e_1\;(a_0\;e_0\;\sigma)$ does not hold and $g_0\;e_0\;\sigma$ does hold.
However, we cannot design any transition that looks at two stores ($a_0\;e_0\;\sigma$ and~$\sigma$) at the same time.

The basic skeleton for the desugaring based on allowing {\it undo\/} actions is the following.
$$\eqalign{
  &x\to z:\; (g_0,a_0) \cr
  &z\to y:\; (g_1,a_1) \cr
  &z\to x:\; (\lnot g_1,{\it undo}(1))
}$$
This, however, also suffers from the serious problem identified above.

Let us take a step back and revisit the semantics.
The earlier description can be recast as pseudo-code.
$$\vbox{
  \settabs\+\quad&\quad&\quad&\quad&\cr
  \+    ${\it Process}(e)$\cr
  \+&     ${\it state}:=({\it start},0)$\cr
  \+&     {\bf while} $e$ is not empty\cr
  \+&&      ${\it candidates}:=[]$\cr
  \+$L:$&&  {\bf for each} transition ${\it vertex}({\it state})\to x:\;
                               (g_0,a_0),\ldots,(g_{n-1},a_{n-1})$\cr
  \+&&&       {\bf if} $n<{\it size}(e)$ {\bf then continue} loop $L$\cr
  \+&&&       $\sigma:={\it store}(\it state)$\cr
  \+&&&       {\bf for} $k\in[\,0.\,.\,n)$ \cr
  \+&&&&        {\bf if} $\lnot(g_k\;e_k\;\sigma)$
                      {\bf then continue} loop $L$\cr
  \+&&&&        $\sigma:=a_k\;e_k\;\sigma$\cr
  \+&&&     append $(x,\sigma,n)$ to {\it candidates}\cr
  \+&&      {\bf if} no {\it candidates} \cr
  \+&&&       delete $e_0$ from $e$\cr
  \+&&      {\bf else}\cr
  \+&&&       $(x,\sigma,n):=\hbox{random candidate}$\cr
  \+&&&       delete $e_0,\ldots,e_{n-1}$ from $e$\cr
  \+&&&       ${\it state}:=(x,\sigma)$\cr
}$$
A {\it state\/} consists of a vertex and a store.
The pseudo-code above refers to the them as ${\it vertex}({\it state})$ and ${\it store}({\it state})$.
I am tempted to begin with the simplest possible implementation:
Use the hook in the program interpreter to build a big list~$e$ and then just implement the above.
Still, it is useful to consider how a better implementation would work.
Indeed, any desugaring (if there is one) corresponds to some implementation.

The main drawback of the simplest implementation is its memory use.
It saves all the events of one execution, but we know that at most two are needed at a time.
(In general, $n$~are needed at a time, where $n$~is the length of the longest transition.
But we only have unit transitions and call--return transitions.)
There is an obvious solution to this problem.
When we process one event $e_k$ we append it to $e$ and then execute the main loop of {\it Process\/} as long as we have enough events in~$e$.
We have enough events in~$e$ when its length is $\ge$ the longest outgoing transition of the current vertex.
When the interpreted program finishes $e$~might not be empty, so we need to run the main loop of {\it Process\/} to exhaust it.
(The $\min$ in the innermost loop is useful only for this last processing.)

% >>>

\bye

% vim:wrap:linebreak:fmr=<<<,>>>:nosi:spell:
